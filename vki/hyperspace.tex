\documentclass[14pt]{beamer}

% Presento style file
\usepackage{config/presento}
% custom command and packages
\input{config/custom-command}
% My macros
/Users/zach/Git/zachs_macros/zachs_macros.tex

% Custom colors
\usepackage{color}
\definecolor{ao}{rgb}{0.0, 0.5, 0.0}

% Configure listings for R
\usepackage{listings}

\lstset{frame=tb,
language=R,
keywordstyle=\color{blue},
otherkeywords={!,!=,~,$,*,\&,\%/\%,\%*\%,\%\%,<-,<<-,\%>\%},
alsoletter={.,_}
}

% For hierarchical lists
\usepackage{outlines}

% Information
\title{LOST IN HYPERSPACE}
\subtitle{\emph{The Curse of Dimensionality}}
\author{Zachary del Rosario}
\institute{zdr@stanford.edu}
\date{September 6th}
%% \date{\today}

% For latex 2018
\makeatletter
\let\@@magyar@captionfix\relax
\makeatother

% --------------------------------------------------
\begin{document}
% --------------------------------------------------
\begin{frame}[plain]
\maketitle
\end{frame}

% --------------------------------------------------
%% SEC: Introduction
% --------------------------------------------------
\begin{frame}[t]{Thought Experiment: Parameter Study}
  Simulation time: one second \\
  Full factorial: $d$ input variables, $10$ points per dimension \\

  \bigskip
  \only<2>{%
    Scaling is \emph{exponential}, i.e.
    \begin{equation*}
      \text{Time} = (10\text{ seconds})^d
    \end{equation*}
  }
\end{frame}

% -------------------------
\begin{frame}{Computational Time}
  \visible<2>{The Curse of Dimensionality}
  \begin{table}
    \begin{tabular}{r|r|l}
    \hline
    Dimension & Time (s) & Comparison\\
    \hline
    1 & 1.0e+01 & Ten seconds\\
    \hline
    5 & 8.6e+04 & One Day\\
    \hline
    10 & 1.0e+10 & Eleven generations\\
    \hline
    18 & 4.3e+17 & Age of Universe\\
    \hline
    20 & 1.0e+20 & 230 x AoU\\
    \hline
    \end{tabular}
  \end{table}
\end{frame}

% -------------------------
\begin{frame}{UQ Tasks}
  High-dimensional integrals
  \begin{itemize}
  \item Moments $\E[\mX^n]$
  \item Probabilities $\E[\i1(\mX<x)]$
  \end{itemize}

  \bigskip Quadrature via full factorial (tensor grid) is exponential \\
  We need to address the Curse
\end{frame}

% -------------------------
\begin{frame}{Goals}
  \begin{itemize}
  \item Motivate
  \item Details and pointers (*)
  \item Shameless self-promotion
  \end{itemize}
\end{frame}

% -------------------------
\begin{frame}{Outline}
  \begin{outline}
  \1 UQ Tasks (Done)
  \1 Curse of Dimensionality
    \2 High-dimensional geometry
  \1 Lifting the Curse
    \2 Dimension reduction
  \end{outline}
\end{frame}

% --------------------------------------------------
%% SEC: Curse of Dimensionality
% --------------------------------------------------
\begin{frame}{Background}
  \begin{outline}
  \1 ``Curse of Dimensionality'' -- Richard Bellman (1961) \\
  \1 \emph{Vague} term
    \2 Integration
    \2 Sampling
    \2 Machine Learning
    \2 Distance
    \2 Big data
  \end{outline}
\end{frame}

% -------------------------
\framecard[colorblue]{{\color{white}%

``The trend today is towards more observations \emph{but even more so},
    \alert{to radically larger numbers of variables} – voracious, automatic,
    systematic collection of hyper-informative detail about each observed
    instance.''

    \bigskip
    -- David Donoho, 2000\\
    \tiny (Emphasis added)

}}

% -------------------------
\begin{frame}{My Point}
  Curse of Dimensionality is \emph{everywhere} \\
  Lots of \emph{very different} perspectives

  \bigskip
  So what's up with high dimensions?
\end{frame}

% -------------------------
\framecard[colorgreen]{{\color{white}\hugetext{%
      \centering%
      Weird Facts\\
      about\\
      High-\\
      Dimensional\\
      Geometry
}}}

% -------------------------
\begin{frame}{Fact 1}
  The hypersphere has vanishing interior
\end{frame}

% -------------------------
\begin{frame}{Unit Hypersphere Volume}
  \begin{equation*} \begin{aligned}
      HV &= \int\cdots\int \alert{r^{d-1}}\,
           T(\varphi_{1},\dots,\varphi_{d-1})\,
           dr d\varphi_{1}\cdots d\varphi_{d-1}
  \end{aligned} \end{equation*}
\end{frame}

% -------------------------
\framepicv[0.8]{../images/surface_density}{}

% -------------------------
\begin{frame}{Fact 2}
  The hypersphere concentrates at \emph{the} equator
\end{frame}

% -------------------------
\framepicv[1.0]{../images/great_circle}{
 \begin{textblock}{7}(0.0,5.7)
    {\tiny By P.wormer}
 \end{textblock}
}

% -------------------------
\begin{frame}{A Wordgame}
  \emph{An} equator is always $d-1$ dimensional \\

  \bigskip An epsilon-band around a great circle can be made to hold an
  arbitrary volume-fraction
\end{frame}

% -------------------------
\framepicv[0.8]{../images/equator}{}

% -------------------------
\begin{frame}{Fact 3}

\end{frame}

\end{document}
